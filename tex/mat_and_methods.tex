\chapter{Materials and Methods}\label{chap:mat_and_methods}
In figure \ref{} a brief overview of the processes applied in this work is displayed. In the coming sections, each compartment will be separately analyzed.

\begin{figure}
	content...
\end{figure}



\section{General}
\section{Data description}
In this study, targeted on humans, a cross section between the dependent cerebral asymmetry and the independent genetic factors is performed, in an effort to discover affiliated genetic regions and provide a novel understanding of the related genes cooperation. With the advent of technology capable to collect and process genomes from different individuals in relatively high speed, vast databases have been constructed. One of the main players in the data collection has been UK Biobank; a large-scale database from a randomized consortium of 500,000 individuals, whose genome has been collected, from whom  48,000 subjects had also participated in brain \ac{mri} collection process, as of December 2020 \cite{Littlejohns2020}. In this thesis, we exploit this newly acquired dataset to identify the key loci that are related to the human brain surface symmetry. Only healthy self-proclaimed white European individuals were considered. A large dataset of 19,654 individuals was used as the main, discovery dataset, while a smaller one, coming from a different batch, of 16,342 individuals was used as a replication dataset during \ac{gwas}.
\section{Phenotype}
\section{\Ac{mri} Shapes Normalization}\label{subsec:shape_normalization}
The current work applies principles from general symmetry studies to model cortical asymmetry. For any of these analyses to occur, the preprocessing of \ac{3d} shapes produced from \ac{mri} scans needs to be considered. \Ac{mri} output is affected by the subject positioning and technical error \cite{Wittens2021}.  Volumetric differences also increase the level of discrepancies among \ac{mri} samples.  To prevent positioning and volume deviations from gravely affecting shape comparisons, a normalization is required\cite{Klingenberg2020}. The samples are represented as a set of landmarks $\mathcal{G}$ of predefined dimensionality, joined together with predefined edges $\mathcal{V}$, forming a multiple-connected structure $S$, that is a graph in which there is at least one path joining any two vertices. The normalization is performed through the application of \ac{gpa}. \Ac{gpa} is an algorithm that iteratively performs translation, scaling and rotation on a given set of structures $S$, given a reference $S_0$, aiming to minimize the euclidean distance of corresponding points. The transformed samples then belong to what it has been coined as Kendall Space \cite{Klingenberg2020}. Under the framework of symmetry analysis, a single hemisphere is considered to be one of the $S$ structures. To apply any symmetry analysis, therefore, one of the individual hemispheres needs to be mirrored on the other side of the midsagittal plane, and then \ac{gpa} is applied to align all hemispheres at once.

\section{Shapes Partitioning} TO BE FIXED / CONTINUED
The present work evaluates the brain asymmetry genetic landscape in a coarse-to-fine segmentation. The technique has been used in a number of different related studies \cite{Claes2018}\cite{Naqvi2021}, yielding results that are in accordance with the underlying anatomic features. The main reason behind this partitioning is the intrinsic complexity of the studied phenotype, eliciting expected differences in the genomic profiles of each cerebral cortex region. The basic assumption made is that topologically close landmarks share similar genetic background. In general though, this type of distance-based clustering is governed by the least quantity of assumptions, regarding the shape or form of the cluster \cite{VonLuxburg2007}. The partitions' genetic juxtaposition is valuable for identifying which regions share similar significant genetic loci, highlighting the corresponding genes contribution, or showcasing the specialization of certain regions that share little to no similarities with their neighbors. Identifying the latter provides a way of mapping the developmental activation of each locus, bringing forth the opportunity to augment the results of related developmental studies \cite{Vijayakumar2016}. 
\Acf{hsc} is an unsupervised method of iterative partitioning, that makes use of the distance matrix eigenvectors \cite{Ng2002}. It results into a binary tree structure (i.e. each parent shape is partitioned into two children). In the current study, a level-4 partitioning is performed, resulting into 31 partitions. Subsequently, they are transformed to the corresponding principal components that explain 80\% of the variance, not only for reasons of further dimensionality reduction, but also to ensure that the resulting traits are orthogonal, and therefore compatible for \ac{ldsc} analyses

\section{Symmetry Statistical analysis}
Bilateral asymmetry is mainly described using three components in literature \cite{klingenberg2002}\cite{Vingerhoets2021}. \Acf{da}, the main focus of this study, corresponds to the hemispheric side effect, namely how the intrinsic (i.e. genetic) properties of the studied population are manifesting across individuals. Antisymmetry, which is related to the effect where sidedness is random in a population (i.e. left-right pattern is mirrored to a right-left pattern), is not observed in the human cerebral cortex, in contrast to other internal organs positions, or organisms \cite{Neubauer2020}. The third component, \acf{fa}, encompasses any random developmental and environmental effects, that cannot be explained with the existing knowledge. The observed deviations can be statistically linearly modeled as products of two effects, the hemisphere side studied and the individual specimen analyzed, as well as their interaction \cite{klingenberg2002}. Formally, based on \cite{VanDongen1999} assuming the presence of replications for each observation per individual, to account for technical error, a mixed linear model representing the aforementioned dependencies is defined as:
\begin{equation}
	Y_{ijk} = \mu + \beta + I_i + S_{ij} + E_{ijk}
\end{equation}
where $Y_{ijk}$ is the phenotype of the i-th individual, from the j-th side, under the k-th replication, $\mu$ and $\beta$ are the fixed intercept and fixed side effect respectively, $I_i\sim\mathcal{N}(0,\sigma^2_{ind})$ is the random individual effect,  $S_{ij}\sim\mathcal{N}(0,\sigma^2_{FA})$ is the random side and individual specific effect, matched to \ac{fa}, and $E_{ijk}\sim\mathcal{N}(0,\sigma^2_{ME})$ is the measurement error. Replications are necessary in such analyses, in order to distinguish \ac{fa} effect from measurement error. Given this definition, a way to measure the statistical significance is performed through an F-test applied on 2-way \ac{anova}, to relate the \ac{rss} ratios of effects to observable error terms, and of fluctuating effect to the measurement error. Extra care needs to be given on the determination of the \ac{dof} of each term, given the preprocessing applied to bring the hemispheres surfaces into Kendall shape space \cite{klingenberg2002}. Given that the analysis is performed on a pair of symmetric objects, and not on a single symmetric object, this configuration is named \textbf{matching asymmetry analysis}.
