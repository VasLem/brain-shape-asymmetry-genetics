\chapter{Discussion}\label{chap:discussion}
\section{Contributions}
In the current study, a detailed  data-driven multi-level analysis statistically elucidated the origins of cortical asymmetry, a complex multivariate phenotypic trait. The degree on which plasticity effect is dispersed throughout the brain was statistically mapped using 2-way \ac{anova} and genetically quantified, through heritability studies. A coarse-to-fine data-driven segmentation identified homogeneously symmetric regions, without any prior anatomic knowledge.  Novel causal region-specific genetic variants were identified after \ac{mvgwas} on the derived partitions, complementing the existing literature \cite{Sha2021}. Different spatially-dependent genetic profiles were identified. Connections with biological pathways, concerning intra- and extra-cellular organization and the formation of symmetry axes, were made, by examining protein-protein interactions. The effect of a strong regulating, spatially dependent epigenetic effect on development was determined. Furthermore, gender-controlled epigenetic modifications appeared to affect cortical asymmetry. Gene- and \ac{snp}-level associative studies  with other genetically-driven traits led to the establishment of a tight genetic connection between  brain shape and asymmetry, while strong evidence was also found for supporting handedness relation with fusiform asymmetry, a finding that correlates with observed asymmetric activity from functional \ac{mri} studies.
\section{Limitations and possible extensions}
Several limitations were detected during the conduction of this study, which could potentially be avoided by a future extended research. As far as the first stage of the analysis is concerned, lack of access to a test-retest dataset from UK Biobank and the disproportionate permutation spaces of 2 way \ac{anova} increased the uncertainty of the credibility of the results. In addition, covariates control prior to the statistical analysis could improve results. In the second stage, the inability to retrieve a signed statistic from the selected \ac{mvgwas} did not allow to deduce whether identified \acp{snp}' presence contributes positively or negative to cortical asymmetry, while also prohibiting the application of \ac{ldsc} during cross-trait analysis. Another stage of multivariate regression, possibly in the form of \ac{cca}, using the lead \acp{snp} as covariates and the phenotype as predicted variable, could produce signed effects for each variant. An additional possible extension would be to repeat the FUMA meta-analysis for the entire set of identified partitions, in order to refine the localization of the analyses. Gender-based studies, that add gender as an additional covariate during  \ac{mvgwas}, could be used in an \ac{anova} setting to derive the degree of the variance explained by gender for each \ac{snp}. Lastly, modeling the cortical asymmetry heritability profile using ideas from statistical shape modeling and identifying, given an individual, what the expected asymmetry is, could provide a measure of the degree of plasticity in a spatially dependent manner, giving rise to a personalized method of defining cortical structure.

