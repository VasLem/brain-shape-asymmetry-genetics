\chapter{Discussion}\label{chap:discussion}

\section{Comparison with literature}
Cortical symmetry is a phenotypic trait exhibiting both global and local properties. It has been extensively anatomically studied in the past as large deviations present great medical value. Situs inversus, the condition where antisymmetry in the organs placement is exhibited for an individual, is not observed in the asymmetric nature of the cortex, indicating the significance of counter-clockwise torque and other global traits in health and survival. Furthermore, the universality of common sulci and gyri asymmetries, such as the one observed for the central sulcus, implicate genetic factors also in a local setting.  \citet{Sha2021} mainly analyze features that have already been extensively studied in the literature, following carefully expert supervised anatomical parcellation of the human cortex, neglecting regions identified to have low heritability, based on the GCTA software, mainly located in the vicinity of the motor cortex and occipital lobe. They identified lead variants by inspecting the meta-analyzed univariate \ac{gwas}, on each DK atlas-specific region, and then traced back the results to each region separately, using phenotypic decomposition\cite{Lin2020}. Therefore, localized analysis was performed only in an indirect manner, with a predefined parcellation. 

On the other hand, the present study followed a data-driven, multi-level analysis, similar to the work presented in \cite{Naqvi2021} and \cite{Claes2018}, focusing on capturing the entire variability of each  \ac{hsc} partition. This led to novel extended findings that also partially validate with greater support the results retrieved from \citet{Sha2021}. The detection of chromosome 15 implication in the formation of cortical asymmetry was not only a premier finding, but also consisted the greatest signal in the \ac{gwas} results for the entire hemisphere. The corresponding lead \ac{snp} resides closest to C15orf53 gene, with a long non-coding \ac{rna} (lncRNA) transcript and a disputed role of being a genetic etiologic factor of Schizophrenia \cite{Kranz2012}. 

Furthermore, substantially greater heritability was detected throughout the studied partitioning levels (maximum 0.22 versus previously reported maximum 0.10 \cite{Sha2021}), explaining missing heritability and providing better estimates for underlying plasticity. At the gene level, significant correlations with cytoskeleton formation, morphogenesis and other prenatal developmental stages were observed, while at a protein-protein interaction level, connections with principal symmetry axes determination were detected. An additional observation relates region-specific differential gene inhibition, due to epigenomics, with cortical asymmetry development. The genetic correlation with handedness comprised an interesting finding, in line with other relevant studies \cite{Kong2021}, whereas correlation with gender-specific enrichment of tissues and traits provided insights into the possible connection of cortical asymmetry with sex, a fact supported by literature, with males exhibiting greater asymmetry than females \cite{Guadalupe2015}. Brain shape trait exhibited strongly similar genetic response \cite{Naqvi2021} across the different partitions, genetically bridging the notion of symmetry with the overall cortical shape. More direct comparisons with related studies were not applicable,  due to the distinct phenotypic segmentation and the scarcity of similar research with published results, connecting genetic factors and cortical symmetry.

\section{Controversies leading to new questions}
\subsection{Dependency on gender}
Given that covarariates control has already been applied and any controllable gender effect has been removed from the analysis, through the ablation of the sex chromosome, before \ac{gwas}, it would have been expected that no gender-specific differences would be detected. However, the results reported in \autoref{sec:res_functional}, especially when superimposed with the observations in the separate analysis of partitioned chromatin accessibility heritability in \autoref{sec:ldsr_results}, point to differential cortical structure characteristics between male and females, caused by distinct epigenetic regulation. The fact that \cite{Sha2021} report a significant lead \ac{snp} on the X chromosome further corroborate such a dependency at the \ac{dna}-replication level.

\subsection{Cross-trait correlation results differ at \acs{snp}- and gene- level}
Although correlation with diseases such as \ac{ad}, \ac{bd} and \ac{asd} was detected at the gene level, in line with other studies \cite{Sha2021,Wang2018,Roe2021}, at the \ac{snp} level this connection was not made, implying that the gene sets intersection may be a false-positive finding. On the other hand, the functional augmentation of the underlying gene sets using GREAT, as well as genes co-regulation and co-expression potentially indirectly relate those diseases with brain shape asymmetry \cite{Siewert-Rocks2022}. Therefore this controversial discovery demands a more extended, set of diffential gene expression analysis, to identify such patterns.

\subsection{Existence of subpopulation structure}
Another controversy was detected related to the existence of subpopulation structure. Although the identification of region-based characteristics, that is the red-hair trait and the skin pigmentation, correlation with cortical asymmetry suggests the existence of subpopulation structures, the intercept value identified during \ac{ldsr} suggests the opposite. This disagreement questions the validity of the assumption of \ac{snp} heritability uniformity made by \ac{ldsr} and calls for a partitioned heritability study on the loci of interest.


\section{Contributions}
In the current study, a detailed  data-driven multi-level analysis statistically elucidated the origins of cortical asymmetry, a complex multivariate phenotypic trait, on healthy individuals of European origin from the largest known \ac{mri} database to date, UK Biobank \cite{Littlejohns2020}. The degree on which plasticity effect is dispersed throughout the brain was statistically mapped using 2-way \ac{anova} and genetically quantified, through heritability studies. A coarse-to-fine data-driven segmentation identified homogeneously symmetric regions, without any prior anatomic knowledge.  Novel causal region-specific genetic variants were identified after \ac{mvgwas} on the derived partitions, complementing the existing literature \cite{Sha2021}. Different spatially-dependent genetic profiles were identified. Connections with biological pathways, concerning intra- and extra-cellular organization and the formation of symmetry axes, were made, by examining protein-protein interactions. The effect of a strong regulating, spatially dependent epigenetic effect on development was determined. Furthermore, gender-controlled epigenetic modifications appeared to affect cortical asymmetry. Gene- and \ac{snp}-level associative studies  with other genetically-driven traits led to the establishment of a tight genetic connection between  brain shape and asymmetry, while strong \ac{snp}-level genetic correlation was detected relatively to intellectual skills, handedness, \ac{ocd} and neuroticism. At the same time, computational acceleration was achieved, without observable loss in accuracy, through the application of simple operations, such as the average shape downsampling discussed in \autoref{sec:methods_on_pheno} that made the statistical analysis feasible, and the mean substitution of the \acp{snp}, that permitted a significant speed up of the \ac{cca} analysis.

\section{Limitations and possible extensions}
\subsection{Limitations}

\subsubsection{First stage}
Several limitations were detected during the conduction of this study, some of which could potentially be avoided by an extended future research. As far as the first stage of the analysis is concerned, lack of access to a test-retest dataset from UK Biobank and the disproportionate permutation spaces of 2-way \ac{anova} increased the error margin of the results, and the total variability of the data was not captured, because of the small sample size used, despite the three experimental iterations aggregation. In addition, covariates control prior to the statistical analysis could potentially provide a more consistent shape normalization and improve results quality. Furthermore, adaptive remeshing, by performing non-rigid group-wise registration, in place of the used naive method, would possibly have increased the results resolution, in exchange for greater computational demands.

\subsubsection{Second stage}
At the second stage, the absence of a signed statistic from \ac{mvgwas} did not allow to deduce whether identified \acp{snp}' presence contributes positively or negatively to cortical asymmetry, while also prohibiting the application of \ac{ldsc} during cross-trait analysis. Another step of multivariate regression, possibly in the form of \ac{cca}, using the lead \acp{snp} as covariates and the phenotype as predicted variable, could produce signed effects for each significant variant. 

Furthermore, 1000G project Phase 3 \ac{snp} filtering greatly reduced the amount of significant genes, with p-values lower than 5e-8, approximately by 82\% for the entire hemisphere \ac{gwas} (table \autoref{tab:1000g_entire_hemisphere} and \autoref{fig:1000g_entire_hemisphere}), while certain partitions lost all the \acp{snp} for at least one chromosome, whose p-value is lower than the Bonferroni threshold of 5e-8/31 (table \autoref{tab:1000g_lost_bonf_chr}). Although UK Biobank makes use of the 1000G Phase 3 project as the reference genome to perform phasing and imputation \cite{Bycroft2018}, in the current study, the high percentage of significant \ac{snp} pruning implies that non-significant \acp{snp} relationships are mostly driving the correlation studies and the LD score analysis.

It is also noteworthy to mention that the initial \ac{snp}-based filtering (cf. \autoref{sub:snpbased_filt}) has pruned away variants, which could potentially have a great impact on the studied phenotypic trait. However, studying the complementary case is difficult and highly error-prone, heavily relying on the sequencing technologies. Including rare variants and indels requires a denser genotyping, feasible through whole genome sequencing (WGS) \cite{Kierczak2022,Cirulli2020}, a technology to-be applied on UK Biobank anticipated dataset update \cite{Halldorsson2022}.


\subsection{Further extensions}
A possible extension would be to expand the gene-based meta-analysis on the entire set of identified partitions (i.e. not only on the second level ones), and develop a hierarchy-dependent \ac{gsea}, to promote cohesive results and remove false positives. Gender-based studies, that add gender as an additional covariate during  \ac{mvgwas}, could be used in a regression setting to derive the degree of variance explained by gender for each gene \cite{Rawlik2016}. Partitioned-based heritability analysis, centered at the regions of interest, could provide an answer to whether localized subpopulation structures exist. Lastly, modeling the cortical asymmetry heritability profile through epistasis analysis and even statistical shape modeling, extending existing ideas \cite{Filipe2019,Claes2014,White2020}, could make an approximate measurement of cortical plasticity in a spatially dependent manner possible, providing grounds for a personalized method of cortical structure anomaly detection and diagnosis of related diseases.


