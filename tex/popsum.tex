
\newenvironment{popsum}%
{\cleardoublepage\thispagestyle{empty}\null\vfill\begin{center}%
		\bfseries Popularized Summary\end{center}}%
{\vfill\null}
\begin{popsum}
The overall purpose of this thesis is to complement the existing bibliography on the detection and examination of the genetic associations of brain shape asymmetry.  Our body generally exhibits bilateral symmetry, namely our sides are mirrored images of each other. However, if someone pays close attention, not one feature is completely symmetric on each side of our body. In this thesis, we seek the genetic roots of this intrinsically complex trait, investigating the brain surface on healthy individuals of European origin from UK Biobank database. Cortical asymmetry has been found to be correlated with a variety of brain-related diseases, such as OCD, neuroticism and bipolar disorder,  and an early detection of the accountable genetic variants could be used for personalized diagnosis and treatment. Furthermore, brain asymmetry does not display the phenomenon of situs-inversus, that is internal organs occurring mirrored at certain individuals, which leads us to believe there is evolutionary pressure on this trait manifestation. Therefore, substantial heritability, namely phenotypic variation explained by genetic changes, is expected.

To this end, asymmetry statistical analysis was performed based on brain MRI scans, detecting regions with high affinity to display genetically inscribed asymmetry. A data-driven approach was then followed, where the brain surface was partitioned in multiple hierarchical levels, without any prior anatomic knowledge. Aggregated asymmetry measurements, retaining most of the included variation, were retrieved from the segments, whose genetic correlation was then examined through a multivariate genetic statistical analysis. Recognized significant variants were then compared against reported traits from other databases, based on monotonicity.  Functional annotations were constructed,  associating genetic variants to genes, offering an insight into the functional reasoning behind the brain shape asymmetry existence. The degree on which plasticity effect (that is the amount of variation not explained by the genome, but by  environmentally affected development) is dispersed throughout the brain was statistically mapped and genetically quantified, through genetic heritability studies. Novel causal region-specific genetic variants and increased heritability was identified throughout the entire hemisphere and the derived partitions, extending the related literature. Different spatially-dependent genetic profiles were identified. Connections with biological pathways, concerning intra- and extra-cellular organization and the formation of biological symmetry axes, were made, by examining protein-protein interactions. The effect of a strongly regulating, spatially dependent epigenetic effect on development was determined. Furthermore, gender-controlled epigenetic modifications appeared to affect cortical asymmetry. Lastly, associative studies led to the establishment of a tight, strongly supported, genetic bridge between  brain shape and asymmetry.
\end{popsum}