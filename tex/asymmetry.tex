\chapter{Asymmetry Phenotypic Analysis}\label{chap:asymmetry}
\section{Statistical Analysis of Asymmetry Components}
Asymmetry is mainly described using three components in literature \cite{klingenberg2002}\cite{Vingerhoets2021}; the directional asymmetry, which is the focus of this study, that corresponds to the hemisphere side effect; the antisymmetry, which is the state of asymmetry in which the sidedness is random in a population (ie. left-right randomly switches to right-left). 

are classified as alternating directional asymmetry across individuals, a phenomenon 
 the fluctuating asymmetry, which is the random effect
The observed deviations can be statistically formulated as products of two fixed effects, the hemisphere side studied and the individual specimen analyzed, as well as their interaction \cite{klingenberg2002}, while individual-specific random effects also need to be considered.
\section{Phenotypic Partitioning}\label{sec:hsc}
Hierarchical spectral clustering is an unsupervised method of iterative partitioning, that makes use of the distance matrix eigenvectors \cite{Ng2002}. It results into a binary tree structure, where each parent shape is partitioned into two children. In the current study, a level-4 partitioning is performed, resulting into 31 partitions. Subsequently they are transformed to the corresponding principal components that explain 80\% of the variance, for reasons of further dimensionality reduction.