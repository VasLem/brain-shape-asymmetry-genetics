\chapter{Introduction}
\label{chap:introduction}

\section{The notion of brain symmetry}
Cerebral bilateral symmetry is a universal quality of organisms belonging to the Bilateria lineage \cite{Corballis2009}\cite{Bailly2013}. For the mammals group, the brain is anatomically divided into a left and right hemisphere. Asymmetric cell division during brain development, initially observed among neuroblasts, causes shape deviations between the two hemispheres (\autoref{fig:brainlat}). This property gives rise to partial functional disassociation, called brain lateralization , with subsets of tasks requiring differential activation of each hemisphere. Lateralization becomes visible when examining organisms' behavior, with the most studied trait in humans being handedness and language \cite{Schmitz2019}. Along with the purely genetic reasons, environment also plays a significant role in affecting cerebral bilateral asymmetry.  Of primary interest in this work is \ac{da}, the asymmetric component that arises by comparing a single individual's hemispheric surfaces landmarks differences, computed through a process of alignment, reflection and subtraction of the landmarks pairs, as discussed in detail in \autoref{chap:asymmetry}. \ac{da} captures information about anatomic characteristics, such as the overall counterclockwise torque, named `Yakovlevian torque' \cite{LeMay1976}, that is observed in humans between the right and left hemisphere (\autoref{fig:yaktorque}). Past studies have shown that abnormal \ac{da} may be an indication of certain diseases. The lack of it may imply schizophrenia predisposition \cite{Ribolsi2014}. Any significant abnormalities may be indicative of other psychiatric disorders, such as autism or developmental language disorder \cite{Herbert2005}\cite{Kong2022}. 

\begin{figure}
	\centering
	\includesvg[width=\textwidth]{da_visualization}\\
	\caption{Illustration of brain asymmetry. Normalized differences of the distances of landmarks from the center of mass of each averaged hemisphere midthickness surface, after a scaling and alignment process, across the studied population.}
	\label{fig:brainlat}
\end{figure}

\section{Data description}
In this study, targeted on humans, a cross section between the dependent cerebral asymmetry and the independent genetic factors is performed, in an effort to discover affiliated genetic regions and provide a novel understanding of the related genes cooperation. \Ac{gwas} have shed light on the correlation between phenotypic traits and genetic content. With the advent of technology capable to collect and process genomes from different individuals in relatively high speed, vast databases have been constructed. One of the main players in the data collection has been UK Biobank; a large-scale database from a randomized consortium of 500,000 individuals, whose genome has been collected, from whom  43,000 subjects had also participated in brain \ac{mri} collection process, as of December 2020. In this thesis, we exploit this newly acquired dataset to identify the key loci that are related to the human brain surface symmetry. Only healthy self-proclaimed white European individuals were considered. The surface analyzed is exclusively the one coined as midthickness, derived as the one half way between the pial and white matter, of the cerebral cortex. More details on the dataset description and the involved data preprocessing can be found in chapters \ref{chap:dataset} and \ref{chap:preprocessing}.

\begin{figure}
	\centering
	\includesvg[]{torque0510}
	\caption{Illustration of the Yakovlevian torque. Displayed by a red line rotated counterclockwise 0.51 degrees in relation to the perfectly vertical black line, as calculated by using the average angle of the longest edges (in blue) of the convex hulls of the horizontal plane projection of each hemisphere midthickness surface, after a scaling and alignment process, across the studied population.}
	\label{fig:yaktorque}
\end{figure}

\section{Breaking the complexity into parts}
The present work evaluates the brain asymmetry genetic landscape in a coarse-to-fine segmentation, through the application of \ac{hsc}\cite{Ng2002}, discussed in \autoref{sec:hsc}. The technique has been used in a number of different related studies \cite{Claes2018}\cite{Naqvi2021}, yielding results that are in accordance with the underlying anatomic features. The main reason behind this partitioning is the intrinsic complexity of the studied phenotype, eliciting expected differences in the genomic profiles of each cerebral cortex region. The basic assumption made is that topologically close landmarks share similar genetic background. In general though, this type of distance-based clustering is governed by the least quantity of assumptions, regarding the shape or form of the cluster \cite{VonLuxburg2007}. The partitions' genetic juxtaposition is valuable for identifying which regions share similar significant genetic loci, highlighting the corresponding genes contribution, or showcasing the specialization of certain regions that share little to no similarities with their neighbors. Identifying the latter provides a way of mapping the developmental activation of each locus, bringing forth the opportunity to augment the results of related developmental studies \cite{Vijayakumar2016}.

\section{Searching for the origin}
The genomic studies are performed under the framework of \ac{snp}-by-\ac{snp} \ac{cca}. The goal is to incorporate multi-allelic \acp{snp} and, more importantly, multivariate phenotype, in a single hypothesis test per \ac{snp}, that is whether the phenotype is significantly correlated with each analyzed \ac{snp}. In general, there is an abundance of strategies on how to perform multivariate \ac{gwas}, ranging from direct methods, that approximate the inputs relation either in an unbiased manner or making certain educated guesses, to more complex techniques, that increase statistical power by transforming the inputs, at the expense of explanatory ability \cite{Galesloot2014}. There are also methods that are based on the meta-analysis of outcomes from univariate studies, commonly used to juxtapose experiments from separate sources, for which the original data is absent or the exact replication of the study is arduous \cite{Cichonska2016}. Which approach performs best mainly lies on the dataset properties and the nature of the scientific question. Factors such as low sample size \cite{Sheng2021}, genes pleiotropic effects \cite{Fernandes2021} or within-study variation \cite{Jackson2011} tend to handicap the statistical modeling and increase the type I and II errors of the corresponding hypothesis tests. In this study, \ac{cca} was primarily chosen due to the high capacity in efficiently reducing the inputs dimensionality while preserving most information regarding their correlation. Diverse experiments, analyzed in \autoref{chap:gwas}, have been applied to identify the method that gives high fidelity results, consistent with relevant literature. The analysis outcome requires further processing, as explained in \autoref{sec:postprocessing}, to account for the main weakness of this method, that it does not consider the \ac{snp}-to-\ac{snp} effect, tackled using as proxy the notion of \ac{ld}, and subsequently to topologically and functionally enhance the filtered findings. Once this additional step has been performed, a cross-traits analysis is applied, described in \autoref{sec:metaanalysis}, where the \ac{da} genetic signature is compared with the signatures of phenotypic traits, analyzed in a similar study \cite{Naqvi2021}, the cerebral and facial shapes.


\section{Novelties based on related literature} 
 Due to the biological importance of cerebral bilateral asymmetry, it is a subject that has been rigorously studied from multiple viewpoints.
 \subsection{Evolution}
 From an evolutionary stand, it is extremely rare for the right conditions to occur, in order for any soft tissue specimen to be preserved, across a considerable amount of time. The only known way is through mineralization \cite{Purnell2018}. This fact renders a mammal's ancestor brain almost impossible to retrieve. Nevertheless, endocranial imprints have been used as a proxy to describe the relationships between hominids and their ancestors \cite{Balzeau2012}\cite{Neubauer2020}. The reason behind this phenotypic delegation is purely practical. The brain size and shape follow the container volume restrictions. The brain sulci (i.e. grooves) and gyri (i.e. bumps) in humans are the result of the tremendous expansion of the cerebral cortex surface area during fetal development, folding and wrinkling in order to fit the skull \cite{F.Bear2016}. Although such studies support the theory of propagating asymmetry among studied individuals, with the most evident signs of \ac{da} in human skulls, little information about the surface shape can be retrieved, as only the convex hull shape of the brain can be delineated from such process. Through the association of brain asymmetry with \acs{dna}, a universal code among organisms, it becomes possible to deploy tools used by evolutionary geneticists, to identify the phylogenetic tree of this complex trait, locating conserved regions among organisms and their predicted divergence in time, under a pleiotropic model \cite{Koch2021}.
 \subsection{Clinical studies}
 